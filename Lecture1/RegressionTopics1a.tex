\documentclass[10pt,ignorenonframetext,]{beamer}
\setbeamertemplate{caption}[numbered]
\setbeamertemplate{caption label separator}{: }
\setbeamercolor{caption name}{fg=normal text.fg}
\beamertemplatenavigationsymbolsempty

%\usepackage{pgfpages}
%\pgfpagesuselayout{2 on 1}[a4paper,border shrink=5mm]


\usepackage{lmodern}
\usepackage{amssymb,amsmath}
\usepackage{ifxetex,ifluatex}
\usepackage{fixltx2e} % provides \textsubscript
\ifnum 0\ifxetex 1\fi\ifluatex 1\fi=0 % if pdftex
\usepackage[T1]{fontenc}
\usepackage[utf8]{inputenc}
\else % if luatex or xelatex
\ifxetex
\usepackage{mathspec}
\else
\usepackage{fontspec}
\fi
\defaultfontfeatures{Ligatures=TeX,Scale=MatchLowercase}
\fi
\usetheme{Madrid}
% use upquote if available, for straight quotes in verbatim environments
\IfFileExists{upquote.sty}{\usepackage{upquote}}{}
% use microtype if available
\IfFileExists{microtype.sty}{%
\usepackage{microtype}
\UseMicrotypeSet[protrusion]{basicmath} % disable protrusion for tt fonts
}{}
\newif\ifbibliography
\usepackage{color}
\usepackage{fancyvrb}
\newcommand{\VerbBar}{|}
\newcommand{\VERB}{\Verb[commandchars=\\\{\}]}
\DefineVerbatimEnvironment{Highlighting}{Verbatim}{commandchars=\\\{\}}
% Add ',fontsize=\small' for more characters per line
\usepackage{framed}
\definecolor{shadecolor}{RGB}{248,248,248}
\newenvironment{Shaded}{\begin{snugshade}}{\end{snugshade}}
\newcommand{\KeywordTok}[1]{\textcolor[rgb]{0.13,0.29,0.53}{\textbf{{#1}}}}
\newcommand{\DataTypeTok}[1]{\textcolor[rgb]{0.13,0.29,0.53}{{#1}}}
\newcommand{\DecValTok}[1]{\textcolor[rgb]{0.00,0.00,0.81}{{#1}}}
\newcommand{\BaseNTok}[1]{\textcolor[rgb]{0.00,0.00,0.81}{{#1}}}
\newcommand{\FloatTok}[1]{\textcolor[rgb]{0.00,0.00,0.81}{{#1}}}
\newcommand{\ConstantTok}[1]{\textcolor[rgb]{0.00,0.00,0.00}{{#1}}}
\newcommand{\CharTok}[1]{\textcolor[rgb]{0.31,0.60,0.02}{{#1}}}
\newcommand{\SpecialCharTok}[1]{\textcolor[rgb]{0.00,0.00,0.00}{{#1}}}
\newcommand{\StringTok}[1]{\textcolor[rgb]{0.31,0.60,0.02}{{#1}}}
\newcommand{\VerbatimStringTok}[1]{\textcolor[rgb]{0.31,0.60,0.02}{{#1}}}
\newcommand{\SpecialStringTok}[1]{\textcolor[rgb]{0.31,0.60,0.02}{{#1}}}
\newcommand{\ImportTok}[1]{{#1}}
\newcommand{\CommentTok}[1]{\textcolor[rgb]{0.56,0.35,0.01}{\textit{{#1}}}}
\newcommand{\DocumentationTok}[1]{\textcolor[rgb]{0.56,0.35,0.01}{\textbf{\textit{{#1}}}}}
\newcommand{\AnnotationTok}[1]{\textcolor[rgb]{0.56,0.35,0.01}{\textbf{\textit{{#1}}}}}
\newcommand{\CommentVarTok}[1]{\textcolor[rgb]{0.56,0.35,0.01}{\textbf{\textit{{#1}}}}}
\newcommand{\OtherTok}[1]{\textcolor[rgb]{0.56,0.35,0.01}{{#1}}}
\newcommand{\FunctionTok}[1]{\textcolor[rgb]{0.00,0.00,0.00}{{#1}}}
\newcommand{\VariableTok}[1]{\textcolor[rgb]{0.00,0.00,0.00}{{#1}}}
\newcommand{\ControlFlowTok}[1]{\textcolor[rgb]{0.13,0.29,0.53}{\textbf{{#1}}}}
\newcommand{\OperatorTok}[1]{\textcolor[rgb]{0.81,0.36,0.00}{\textbf{{#1}}}}
\newcommand{\BuiltInTok}[1]{{#1}}
\newcommand{\ExtensionTok}[1]{{#1}}
\newcommand{\PreprocessorTok}[1]{\textcolor[rgb]{0.56,0.35,0.01}{\textit{{#1}}}}
\newcommand{\AttributeTok}[1]{\textcolor[rgb]{0.77,0.63,0.00}{{#1}}}
\newcommand{\RegionMarkerTok}[1]{{#1}}
\newcommand{\InformationTok}[1]{\textcolor[rgb]{0.56,0.35,0.01}{\textbf{\textit{{#1}}}}}
\newcommand{\WarningTok}[1]{\textcolor[rgb]{0.56,0.35,0.01}{\textbf{\textit{{#1}}}}}
\newcommand{\AlertTok}[1]{\textcolor[rgb]{0.94,0.16,0.16}{{#1}}}
\newcommand{\ErrorTok}[1]{\textcolor[rgb]{0.64,0.00,0.00}{\textbf{{#1}}}}
\newcommand{\NormalTok}[1]{{#1}}
\usepackage{longtable,booktabs}
\usepackage{caption}
% These lines are needed to make table captions work with longtable:
\makeatletter
\def\fnum@table{\tablename~\thetable}
\makeatother
\usepackage{graphicx,grffile}
\makeatletter
\def\maxwidth{\ifdim\Gin@nat@width>\linewidth\linewidth\else\Gin@nat@width\fi}
\def\maxheight{\ifdim\Gin@nat@height>\textheight0.8\textheight\else\Gin@nat@height\fi}
\makeatother
% Scale images if necessary, so that they will not overflow the page
% margins by default, and it is still possible to overwrite the defaults
% using explicit options in \includegraphics[width, height, ...]{}
\setkeys{Gin}{width=\maxwidth,height=\maxheight,keepaspectratio}

% Prevent slide breaks in the middle of a paragraph:
\widowpenalties 1 10000
\raggedbottom

\AtBeginPart{
\let\insertpartnumber\relax
\let\partname\relax
\frame{\partpage}
}
\AtBeginSection{
\ifbibliography
\else
\let\insertsectionnumber\relax
\let\sectionname\relax
\frame{\sectionpage}
\fi
}
\AtBeginSubsection{
\let\insertsubsectionnumber\relax
\let\subsectionname\relax
\frame{\subsectionpage}
}

\setlength{\parindent}{0pt}
\setlength{\parskip}{6pt plus 2pt minus 1pt}
\setlength{\emergencystretch}{3em}  % prevent overfull lines
\providecommand{\tightlist}{%
\setlength{\itemsep}{0pt}\setlength{\parskip}{0pt}}
\setcounter{secnumdepth}{0}

\title{Further topics in linear regression}
\subtitle{Part 1}
\author{David Barron}
\date{Hilary Term 2017}

\begin{document}
\frame{\titlepage}

\begin{frame}
\tableofcontents[hideallsubsections]
\end{frame}

\section{Introduction}\label{introduction}

\begin{frame}{Introduction}

\begin{itemize}
\tightlist
\item
  Introduction to R
\item
  Review of multiple regression
\item
  Modelling

  \begin{itemize}
  \tightlist
  \item
    Dummy variables
  \item
    Interactions
  \end{itemize}
\item
  Regression diagnostics

  \begin{itemize}
  \tightlist
  \item
    Normality of residuals
  \item
    Collinearity
  \item
    Outliers
  \item
    Heteroskedasticity
  \item
    Linearity
  \end{itemize}
\item
  Sample selection bias
\end{itemize}

\end{frame}

\begin{frame}{Regression model}

\[
y_i = \beta_0 + \beta_1 x_{1i} + \beta_2 x_{2i}+\dots+\beta_k x_{ki}
+ \epsilon_i,
\]

for \(i=1,\dots,n\) sampled observations.
\(\epsilon_i\sim \text{NID}(0,\sigma^2)\).

\textbf{Fitted model}

\[
\hat{y}_i = b_0 + b_1 x_{1i} + b_2 x_{2i} + \dots + b_k x_{ki}
+ e_i = \hat{y}_i + e_i,
\]

where \(b_j\) are estimates of the corresponding \(\beta_j\), and the
\(e_i\) are residuals.

\end{frame}

\section{Modelling}\label{modelling}

\subsection{Dummy Variables}\label{dummy-variables}

\begin{frame}{What are dummy variables?}

Often we want to use explanatory variables in regressions that are
categorical. To do this, we have to use \emph{dummy variables}. Which
category a particular observation falls in to is identified by a series
of binary (0/1) variables, one fewer variables than there are
categories. That's because there is always one category that does not
give us any additional information: if someone isn't a man, they must be
a woman and hence we only need a variable identifying whether someone is
a man (dummy variable = 1) or isn't a man (dummy variable = 0). How,
though, do we interpret the parameter estimates associated with dummy
variables?

\end{frame}

\begin{frame}{Simulated data}

In this example, we have a categorical variable with 4 categories and a
continuous variable that are related to a dependent variable in the
following way.

\[
y = -.7 x_1 - .2 x_2 + .3 x_3 + .9 x_4 + .4 x_c + \epsilon(0,2)
\]

We first perform a regression of \(y\) on the continuous variable,
\(x_c\), only.

\end{frame}

\begin{frame}[fragile]{Regression with continuous variable only}

\begin{verbatim}

Call:
lm(formula = y ~ xc, data = dta)

Residuals:
   Min     1Q Median     3Q    Max
 -6.15  -1.47   0.01   1.37   6.66

Coefficients:
            Estimate Std. Error t value Pr(>|t|)
(Intercept)   0.1095     0.0648    1.69    0.092
xc            0.3753     0.0126   29.73   <2e-16

Residual standard error: 2.05 on 998 degrees of freedom
Multiple R-squared:  0.47,  Adjusted R-squared:  0.469
F-statistic:  884 on 1 and 998 DF,  p-value: <2e-16
\end{verbatim}

\end{frame}

\begin{frame}[fragile]{First category excluded}

\begin{verbatim}

Call:
lm(formula = y ~ xfac + xc, data = dta)

Residuals:
   Min     1Q Median     3Q    Max
-5.270 -1.399  0.013  1.361  6.473

Coefficients:
            Estimate Std. Error t value Pr(>|t|)
(Intercept)  -0.8110     0.1233   -6.58  7.7e-11
xfacB         0.6967     0.1743    4.00  6.9e-05
xfacC         1.2551     0.1743    7.20  1.2e-12
xfacD         1.7307     0.1748    9.90  < 2e-16
xc            0.3852     0.0121   31.94  < 2e-16

Residual standard error: 1.95 on 995 degrees of freedom
Multiple R-squared:  0.522, Adjusted R-squared:  0.52
F-statistic:  272 on 4 and 995 DF,  p-value: <2e-16
\end{verbatim}

\end{frame}

\begin{frame}[fragile]{Last category excluded}

\begin{verbatim}

Call:
lm(formula = y ~ xfac + xc, data = dta)

Residuals:
   Min     1Q Median     3Q    Max
-5.270 -1.399  0.013  1.361  6.473

Coefficients:
            Estimate Std. Error t value Pr(>|t|)
(Intercept)   0.9198     0.1236    7.44  2.2e-13
xfacA        -1.7307     0.1748   -9.90  < 2e-16
xfacB        -1.0340     0.1750   -5.91  4.7e-09
xfacC        -0.4756     0.1745   -2.73   0.0065
xc            0.3852     0.0121   31.94  < 2e-16

Residual standard error: 1.95 on 995 degrees of freedom
Multiple R-squared:  0.522, Adjusted R-squared:  0.52
F-statistic:  272 on 4 and 995 DF,  p-value: <2e-16
\end{verbatim}

\end{frame}

\begin{frame}{What is the relationship between the two?}

\begin{longtable}[]{@{}lll@{}}
\toprule
Category & \(A\) excluded & \(D\) excluded\tabularnewline
\midrule
\endhead
A & \(-0.81\) & \(0.92 - 1.73 = -0.81\)\tabularnewline
B & \(-0.81 + 0.70 = -0.11\) & \(0.92 - 1.03 = -0.11\)\tabularnewline
C & \(-0.81 + 1.26 = 0.44\) & \(0.92 - 0.48 = 0.44\)\tabularnewline
D & \(-0.81 + 1.73 = .92\) & \(0.92\)\tabularnewline
\bottomrule
\end{longtable}

Parameter estimates give how much that category differs from the
\emph{excluded category}.

\end{frame}

\begin{frame}{Interpreting t-values}

Because parameter estimates depend on the arbitrary choice of excluded
category, you can't interpret the \(t-\)values associated with each
estimate in the usual way. To determine whether a dummy variable is
statistically signficant, it is conventional to use an \(F-\) test,
using the formula:

\[
\frac{(RSS_r - RSS_c)/p}{RSS_c/(n-k-p-1)},
\] where \(RSS_r\) is the residual sum of squares (RSS) from the
regression without dummy variables, \(RSS_c\) is the SSR from the
complete model, \(p\) is the number of extra parameters in the complete
model, \(n\) is the sample size, \(k\) is the number of variable in the
restricted model (not counting the constant).

\end{frame}

\begin{frame}[fragile]{Example}

We can get the numbers we want by using the \texttt{anova} function in
\texttt{R}.

\begin{Shaded}
\begin{Highlighting}[]
\KeywordTok{anova}\NormalTok{(cont, xf1)}
\end{Highlighting}
\end{Shaded}

\begin{longtable}[]{@{}rrrrrr@{}}
\toprule
Res.Df & RSS & Df & Sum of Sq & F & Pr(\textgreater{}F)\tabularnewline
\midrule
\endhead
998 & 4192 & NA & NA & NA & NA\tabularnewline
995 & 3778 & 3 & 414 & 36.3 & 0\tabularnewline
\bottomrule
\end{longtable}

You can simply read off the test from this, but you might want to check
the formula above using the RSS numbers.

\end{frame}

\begin{frame}{Duncan's occupational prestige data}

This example uses data on occupational prestige. The outcome variable is
the percentage of survey respondents who rated an occupation's prestige
\textbf{excellent} or \textbf{good}. The explanatory variables are
\emph{income}, which is the percentage of males in the occupation
earning \$3500 or more in 1950; \emph{education}, the percentage of
males in the occupation in 1950 who were high school graduates; and
\emph{type}, which is a factor distinguishing occupations that are
\emph{professions}, \emph{white collar} or \emph{blue collar}.

\end{frame}

\begin{frame}[fragile]{Regression output}

\begin{verbatim}

Call:
lm(formula = prestige ~ income + education + type, data = Duncan)

Residuals:
   Min     1Q Median     3Q    Max
-14.89  -5.74  -1.75   5.44  28.97

Coefficients:
            Estimate Std. Error t value Pr(>|t|)
(Intercept)  -0.1850     3.7138   -0.05   0.9605
income        0.5975     0.0894    6.69  5.1e-08
education     0.3453     0.1136    3.04   0.0042
typeprof     16.6575     6.9930    2.38   0.0221
typewc      -14.6611     6.1088   -2.40   0.0211

Residual standard error: 9.74 on 40 degrees of freedom
Multiple R-squared:  0.913, Adjusted R-squared:  0.904
F-statistic:  105 on 4 and 40 DF,  p-value: <2e-16
\end{verbatim}

\end{frame}

\begin{frame}{Effect plot}

\includegraphics{RegressionTopics1a_files/figure-beamer/unnamed-chunk-6-1.pdf}

\end{frame}

\begin{frame}{Interpretation}

You can see that dummy variables have the effect of shifting estimated
regression lines up or down. The lines are parallel to each other. Here
we can see that occupational prestige increases with education and that
at all levels of education, estimated prestige is lowest for white
collar jobs and highest for professional occupations.

\end{frame}

\subsection{Interactions}\label{interactions}

\begin{frame}{Motivation}

The standard linear regression model implies that the size of the effect
of any given explanatory variable on the outcome variable is the same at
all values of the other explanatory variables. What do we do if we think
that is not true? For example, the effect of marital status and number
of children on wages may be different for men and women. The standard
way of incorporating such interactions is to multiply two variables
together:

\[
Y_i = \beta_0 + \beta_1 X_{1i} + \beta_2 X_{2i} + \beta_3 X_{1i} X_{2i} + \epsilon_i
\]

\end{frame}

\begin{frame}{Example: Labour Force Survey data}

Data from the UK Labour Force Survey gives information about wages as
well as age, gender, marital status and number of children. Wages are
tranformed to an hourly basis, and then logged because otherwise they
would be very skewed.

\includegraphics[width=0.4\linewidth]{RegressionTopics1a_files/figure-beamer/unnamed-chunk-7-1}
\includegraphics[width=0.4\linewidth]{RegressionTopics1a_files/figure-beamer/unnamed-chunk-7-2}

\end{frame}

\begin{frame}[fragile]{Results without interactions}

\footnotesize

\begin{verbatim}

Call:
lm(formula = Loghourpay ~ sex + age + allchildren + married,
    data = lfs)

Residuals:
   Min     1Q Median     3Q    Max
-5.111 -0.377 -0.039  0.362  3.012

Coefficients:
             Estimate Std. Error t value Pr(>|t|)
(Intercept)  1.878155   0.012081  155.46   <2e-16
sexfemale   -0.252642   0.006236  -40.52   <2e-16
age          0.007007   0.000306   22.92   <2e-16
allchildren  0.009708   0.003519    2.76   0.0058
marriedyes   0.113604   0.007628   14.89   <2e-16

Residual standard error: 0.552 on 31405 degrees of freedom
  (32149 observations deleted due to missingness)
Multiple R-squared:  0.0932,    Adjusted R-squared:  0.0931
F-statistic:  807 on 4 and 31405 DF,  p-value: <2e-16
\end{verbatim}

All of these estimates are statistically signficant, although the
overall model fit is pitiful!

\end{frame}

\begin{frame}[fragile]{Results with interactions}

\footnotesize

\begin{verbatim}

Call:
lm(formula = Loghourpay ~ sex + age + allchildren + married +
    sex:married + sex:allchildren, data = lfs)

Residuals:
   Min     1Q Median     3Q    Max
-5.070 -0.372 -0.041  0.356  2.964

Coefficients:
                       Estimate Std. Error t value Pr(>|t|)
(Intercept)            1.814050   0.012471  145.47  < 2e-16
sexfemale             -0.099962   0.010108   -9.89  < 2e-16
age                    0.006595   0.000305   21.64  < 2e-16
allchildren            0.027676   0.004841    5.72  1.1e-08
marriedyes             0.225938   0.010467   21.59  < 2e-16
sexfemale:marriedyes  -0.206089   0.013061  -15.78  < 2e-16
sexfemale:allchildren -0.044640   0.006665   -6.70  2.2e-11

Residual standard error: 0.549 on 31403 degrees of freedom
  (32149 observations deleted due to missingness)
Multiple R-squared:  0.104, Adjusted R-squared:  0.104
F-statistic:  605 on 6 and 31403 DF,  p-value: <2e-16
\end{verbatim}

\end{frame}

\begin{frame}{Interpretation of interactions}

There are two additional parameter estimates, representing the
interaction of sex and marital status, and sex and number of children,
respectively. We can work out the effect of being married on log hourly
wages for men and women as follows:

\begin{longtable}[]{@{}lll@{}}
\toprule
Category & No interaction & With interaction\tabularnewline
\midrule
\endhead
Unmarried men & 1.878 & 1.814\tabularnewline
Married men & 1.992 & 2.04\tabularnewline
Unmarried women & 1.626 & 1.714\tabularnewline
Married women & 1.739 & 1.734\tabularnewline
\bottomrule
\end{longtable}

You can see that in the first column the difference between being
married and unmarried is the same for men and women, but in the second
column the differences are much bigger for men than for women.

\end{frame}

\begin{frame}{Interpretation of interactions 2}

\includegraphics{RegressionTopics1a_files/figure-beamer/unnamed-chunk-11-1.pdf}

\end{frame}

\section{Regression diagnostics}\label{regression-diagnostics}

\subsection{Normality of residuals}\label{normality-of-residuals}

\begin{frame}{Assumption}

The standard assumption of linear regression is that the errors are
normally distributed. If they are not, you will still get unbiased
estimates of the regression parameters. However, the estimates will not
(necessarily) be as efficient as they could be (i.e., standard errors
will be larger than they need to be). Hypothesis testing (which relies
on us knowing the sampling distribution of estimates) also depends on
normality assumption being met.

\end{frame}

\begin{frame}[fragile]{Example}

As an example, look at the Labour Force Survey data again but do the
regression without taking logs of hourly pay.

\scriptsize

\begin{verbatim}

Call:
lm(formula = hourpay0 ~ sex * married + age + sex * allchildren,
    data = lfs)

Residuals:
   Min     1Q Median     3Q    Max
-13.05  -3.79  -1.64   2.11 192.34

Coefficients:
                      Estimate Std. Error t value Pr(>|t|)
(Intercept)            6.91723    0.15278   45.27  < 2e-16
sexfemale             -1.02923    0.12383   -8.31  < 2e-16
marriedyes             2.34207    0.12823   18.26  < 2e-16
age                    0.06402    0.00373   17.14  < 2e-16
allchildren            0.38433    0.05931    6.48  9.3e-11
sexfemale:marriedyes  -2.34017    0.16001  -14.62  < 2e-16
sexfemale:allchildren -0.43559    0.08166   -5.33  9.7e-08

Residual standard error: 6.73 on 31403 degrees of freedom
  (32149 observations deleted due to missingness)
Multiple R-squared:  0.0764,    Adjusted R-squared:  0.0763
F-statistic:  433 on 6 and 31403 DF,  p-value: <2e-16
\end{verbatim}

\end{frame}

\begin{frame}{Density plot}

A density plot is a kind of ``continuous histogram''. You can compare
the distribution of residuals with a normal distribution with the same
mean and standard deviation.

\includegraphics[width=0.45\linewidth]{RegressionTopics1a_files/figure-beamer/unnamed-chunk-13-1}
\includegraphics[width=0.45\linewidth]{RegressionTopics1a_files/figure-beamer/unnamed-chunk-13-2}

\end{frame}

\begin{frame}{QQ-plot}

\includegraphics[width=0.45\linewidth]{RegressionTopics1a_files/figure-beamer/unnamed-chunk-14-1}
\includegraphics[width=0.45\linewidth]{RegressionTopics1a_files/figure-beamer/unnamed-chunk-14-2}

\end{frame}

\subsection{Multi-collinearity}\label{multi-collinearity}

\begin{frame}{Definition}

(Multi-)collinearity is the problem of two or more explanatory variables
not being independent of each other. Strictly speaking, this is not a
violation of the assumptions of the linear regression model, but when
collinearity becomes very high, estimated standard errors become very
high and in some circumstances regression parameter estimates can be
difficult to obtain. One way to measure collinearity relies on
\(R^2_i\), the proportion of the variance of the \(i\) th explanatory
variable that is associated with the other explanatory variables in the
model. That is, if the regression model is \[
y = b_0 + b_1 x_1 + b_2 x_x + \dots + b_k x_k + e,
\] then we regress one explanatory variable on the others: \[
x_1 = c_0 + c_2 x_2 + \dots + c_k x_k + e
\] and find the \(R^2\) of this second regression.

\end{frame}

\begin{frame}{Variance inflation factor and tolerance}

More commonly, two statistics that are derived from \(R^2_i\) are
reported.

\begin{itemize}
\tightlist
\item
  \textbf{Tolerance} \(= (1 - R_i^2);\)
\item
  \textbf{Variance inflation factor} \(= 1 / (1 - R^2_i).\)
\end{itemize}

The VIF is thus the reciprocal of the tolerance. The VIF (or its square
root) is the most commonly reported statistic because it is the impact
on the estimated variance (or standard error) of parameter estimates
that we are usually most concerned about: \[
\sigma^2(b_i) = \frac{\sigma^2_\epsilon}{\sum x_i^2} \times \text{VIF}
\]

A common rule of thumb is that a VIF of 10 or above is a source of
concern. However, treat such rules with caution, as it is possible to
make matters worse by using common ``solutions.''

\end{frame}

\begin{frame}[fragile]{Example}

Calculate the VIF for the Labour Force Survey regression above:

\begin{verbatim}
                 VIF
sex             2.66
age             1.40
allchildren     2.24
married         2.75
sex:married     3.74
sex:allchildren 2.61
\end{verbatim}

You can see that all these VIFs are quite small, so (despite there being
two interaction effects, where collinearity can sometimes be a problem),
we don't have any concerns about this. What do we do if there is
evidence of high collinearity, though?

\end{frame}

\begin{frame}{Solutions?}

In many cases, there is no straightforward solution; if variables are
highly collinear, that's just the way the world is and you can't change
it no matter how inconvenient it may be. For example, it might be
difficult to separate the impact of age and years of experience on
wages. It is increasing the risk of failing to reject a null hypothesis
even if it is false, so if estimates are significant anyway, you're OK.
If you need to reduce the impact of collinearity, there are a few
possibilities.

\begin{itemize}
\tightlist
\item
  Collect more data. This reduces standard errors, but it may not be
  practical.
\item
  Combine two or more explanatory variables into a single indicator.
  Only an option in (rare) cases where this would make theoretical
  sense.
\item
  Remove one or more variables from the regression. This is very risky,
  and introduces the broader question of how to select the ``best''
  regression model.
\end{itemize}

\end{frame}

\begin{frame}[fragile]{Misspecification bias}

\begin{verbatim}

Call:
lm(formula = y ~ x1 + x2)

Residuals:
    Min      1Q  Median      3Q     Max
-1.9440 -0.6652 -0.0507  0.7705  1.8246

Coefficients:
            Estimate Std. Error t value Pr(>|t|)
(Intercept)    3.725      0.414    8.99  1.3e-09
x1             0.280      0.173    1.62     0.12
x2            -1.080      0.177   -6.11  1.6e-06

Residual standard error: 1.02 on 27 degrees of freedom
Multiple R-squared:  0.746, Adjusted R-squared:  0.727
F-statistic: 39.6 on 2 and 27 DF,  p-value: 9.3e-09
\end{verbatim}

\end{frame}

\begin{frame}[fragile]

\begin{verbatim}

Call:
lm(formula = y ~ x2)

Residuals:
   Min     1Q Median     3Q    Max
-2.437 -0.577  0.138  0.677  1.792

Coefficients:
            Estimate Std. Error t value Pr(>|t|)
(Intercept)   3.3458     0.3514    9.52  2.8e-10
x2           -0.8393     0.0986   -8.51  3.0e-09

Residual standard error: 1.05 on 28 degrees of freedom
Multiple R-squared:  0.721, Adjusted R-squared:  0.711
F-statistic: 72.4 on 1 and 28 DF,  p-value: 2.99e-09
\end{verbatim}

\end{frame}

\begin{frame}{Note}

This is based on simulated data with \(b_0=4\), \(b_1=0.5\) and
\(b_2 = -1.3\). The two explanatory variables are strongly correlated.
Removing \(x_1\) from the analysis because it is not statistically
significant introduces bias in the estimate of \(b_2\).

\end{frame}

\begin{frame}{Things to look out for}

\begin{itemize}
\tightlist
\item
  Large change in the parameter estimate of \(b_2\) across the two
  regressions.
\item
  Large change in the \(R^2\) across the two regressions.
\item
  \textbf{Most important} is your theory; make decisions based on
  theory, not by blindly following some statistical ``rule.''
\item
  Consider using one of the step-wise regression methods as an aid to
  model building.

  \begin{itemize}
  \tightlist
  \item
    These are particularly appropriate when you are building models with
    the primary purpose of prediction
  \end{itemize}
\end{itemize}

\end{frame}

\begin{frame}{Stepwise regression}

The basic idea of stepwise regression is to identify a subset of
potential explanatory variables that explain as much variance as
possible in the outcome variable as parsimoniously as possible. There
are two possible approaches:

\begin{itemize}
\tightlist
\item
  We start with a minimal model and add variables until there is no
  improvement in fit;
\item
  We start with all possible variables and remove them until there is no
  deterioration in fit.
\end{itemize}

The criterion most commonly used to assess fit is the Akaike information
criterion (AIC), which is smaller the better fitting the model, taking
account of the number of parameters being estimated.

\end{frame}

\begin{frame}[fragile]{Example}

Data on credit histories of 1,319 applicants for credit cards. The
outcome variable is the number of major negative reports. \emph{Age} in
years; \emph{Income} in US dollars/10,000; \emph{Share} is ratio of
monthly credit card expenditure to yearly income; \emph{Owner} is a
factor, whether a home owner; \emph{Dependents} is number of dependents;
\emph{Months} at current address.

\tiny

\begin{verbatim}

Call:
lm(formula = reports ~ age + income + share + owner + dependents +
    months, data = CreditCard)

Residuals:
   Min     1Q Median     3Q    Max
-1.027 -0.575 -0.415 -0.074 13.416

Coefficients:
             Estimate Std. Error t value Pr(>|t|)
(Intercept)  0.474032   0.139941    3.39  0.00073
age          0.004181   0.004345    0.96  0.33617
income       0.006233   0.024055    0.26  0.79557
share       -2.158597   0.389605   -5.54  3.6e-08
owneryes    -0.238853   0.083708   -2.85  0.00439
dependents   0.024947   0.031925    0.78  0.43469
months       0.000929   0.000617    1.50  0.13270

Residual standard error: 1.33 on 1312 degrees of freedom
Multiple R-squared:  0.033, Adjusted R-squared:  0.0286
F-statistic: 7.47 on 6 and 1312 DF,  p-value: 6.88e-08
\end{verbatim}

\end{frame}

\begin{frame}[fragile]{Backwards elimination}

\begin{verbatim}

Call:
lm(formula = reports ~ share + owner + months, data = CreditCard)

Residuals:
   Min     1Q Median     3Q    Max
-1.049 -0.568 -0.420 -0.088 13.422

Coefficients:
             Estimate Std. Error t value Pr(>|t|)
(Intercept)  0.628971   0.061066   10.30  < 2e-16
share       -2.230742   0.386298   -5.77  9.6e-09
owneryes    -0.188593   0.075701   -2.49    0.013
months       0.001155   0.000568    2.03    0.042

Residual standard error: 1.33 on 1315 degrees of freedom
Multiple R-squared:  0.0315,    Adjusted R-squared:  0.0293
F-statistic: 14.2 on 3 and 1315 DF,  p-value: 3.88e-09
\end{verbatim}

\end{frame}

\begin{frame}[fragile]{Forwards addition}

\scriptsize

\begin{verbatim}

Call:
lm(formula = reports ~ share + owner + months, data = CreditCard)

Residuals:
   Min     1Q Median     3Q    Max
-1.049 -0.568 -0.420 -0.088 13.422

Coefficients:
             Estimate Std. Error t value Pr(>|t|)
(Intercept)  0.628971   0.061066   10.30  < 2e-16
share       -2.230742   0.386298   -5.77  9.6e-09
owneryes    -0.188593   0.075701   -2.49    0.013
months       0.001155   0.000568    2.03    0.042

Residual standard error: 1.33 on 1315 degrees of freedom
Multiple R-squared:  0.0315,    Adjusted R-squared:  0.0293
F-statistic: 14.2 on 3 and 1315 DF,  p-value: 3.88e-09
\end{verbatim}

\normalsize
In this case, both methods give the same answer, which adds to our
confidence. This isn't always the case. Care needs to be taken using
stepwise methods; there is no substitute for thinking!

\end{frame}

\end{document}
