\documentclass[10pt,ignorenonframetext,]{beamer}
\setbeamertemplate{caption}[numbered]
\setbeamertemplate{caption label separator}{: }
\setbeamercolor{caption name}{fg=normal text.fg}
\beamertemplatenavigationsymbolsempty
\usepackage{lmodern}
\usepackage{amssymb,amsmath}
\usepackage{ifxetex,ifluatex}
\usepackage{fixltx2e} % provides \textsubscript
\ifnum 0\ifxetex 1\fi\ifluatex 1\fi=0 % if pdftex
\usepackage[T1]{fontenc}
\usepackage[utf8]{inputenc}
\else % if luatex or xelatex
\ifxetex
\usepackage{mathspec}
\else
\usepackage{fontspec}
\fi
\defaultfontfeatures{Ligatures=TeX,Scale=MatchLowercase}
\fi
\usetheme{Madrid}
% use upquote if available, for straight quotes in verbatim environments
\IfFileExists{upquote.sty}{\usepackage{upquote}}{}
% use microtype if available
\IfFileExists{microtype.sty}{%
\usepackage{microtype}
\UseMicrotypeSet[protrusion]{basicmath} % disable protrusion for tt fonts
}{}
\newif\ifbibliography
\usepackage{color}
\usepackage{fancyvrb}
\newcommand{\VerbBar}{|}
\newcommand{\VERB}{\Verb[commandchars=\\\{\}]}
\DefineVerbatimEnvironment{Highlighting}{Verbatim}{commandchars=\\\{\}}
% Add ',fontsize=\small' for more characters per line
\usepackage{framed}
\definecolor{shadecolor}{RGB}{248,248,248}
\newenvironment{Shaded}{\begin{snugshade}}{\end{snugshade}}
\newcommand{\KeywordTok}[1]{\textcolor[rgb]{0.13,0.29,0.53}{\textbf{{#1}}}}
\newcommand{\DataTypeTok}[1]{\textcolor[rgb]{0.13,0.29,0.53}{{#1}}}
\newcommand{\DecValTok}[1]{\textcolor[rgb]{0.00,0.00,0.81}{{#1}}}
\newcommand{\BaseNTok}[1]{\textcolor[rgb]{0.00,0.00,0.81}{{#1}}}
\newcommand{\FloatTok}[1]{\textcolor[rgb]{0.00,0.00,0.81}{{#1}}}
\newcommand{\ConstantTok}[1]{\textcolor[rgb]{0.00,0.00,0.00}{{#1}}}
\newcommand{\CharTok}[1]{\textcolor[rgb]{0.31,0.60,0.02}{{#1}}}
\newcommand{\SpecialCharTok}[1]{\textcolor[rgb]{0.00,0.00,0.00}{{#1}}}
\newcommand{\StringTok}[1]{\textcolor[rgb]{0.31,0.60,0.02}{{#1}}}
\newcommand{\VerbatimStringTok}[1]{\textcolor[rgb]{0.31,0.60,0.02}{{#1}}}
\newcommand{\SpecialStringTok}[1]{\textcolor[rgb]{0.31,0.60,0.02}{{#1}}}
\newcommand{\ImportTok}[1]{{#1}}
\newcommand{\CommentTok}[1]{\textcolor[rgb]{0.56,0.35,0.01}{\textit{{#1}}}}
\newcommand{\DocumentationTok}[1]{\textcolor[rgb]{0.56,0.35,0.01}{\textbf{\textit{{#1}}}}}
\newcommand{\AnnotationTok}[1]{\textcolor[rgb]{0.56,0.35,0.01}{\textbf{\textit{{#1}}}}}
\newcommand{\CommentVarTok}[1]{\textcolor[rgb]{0.56,0.35,0.01}{\textbf{\textit{{#1}}}}}
\newcommand{\OtherTok}[1]{\textcolor[rgb]{0.56,0.35,0.01}{{#1}}}
\newcommand{\FunctionTok}[1]{\textcolor[rgb]{0.00,0.00,0.00}{{#1}}}
\newcommand{\VariableTok}[1]{\textcolor[rgb]{0.00,0.00,0.00}{{#1}}}
\newcommand{\ControlFlowTok}[1]{\textcolor[rgb]{0.13,0.29,0.53}{\textbf{{#1}}}}
\newcommand{\OperatorTok}[1]{\textcolor[rgb]{0.81,0.36,0.00}{\textbf{{#1}}}}
\newcommand{\BuiltInTok}[1]{{#1}}
\newcommand{\ExtensionTok}[1]{{#1}}
\newcommand{\PreprocessorTok}[1]{\textcolor[rgb]{0.56,0.35,0.01}{\textit{{#1}}}}
\newcommand{\AttributeTok}[1]{\textcolor[rgb]{0.77,0.63,0.00}{{#1}}}
\newcommand{\RegionMarkerTok}[1]{{#1}}
\newcommand{\InformationTok}[1]{\textcolor[rgb]{0.56,0.35,0.01}{\textbf{\textit{{#1}}}}}
\newcommand{\WarningTok}[1]{\textcolor[rgb]{0.56,0.35,0.01}{\textbf{\textit{{#1}}}}}
\newcommand{\AlertTok}[1]{\textcolor[rgb]{0.94,0.16,0.16}{{#1}}}
\newcommand{\ErrorTok}[1]{\textcolor[rgb]{0.64,0.00,0.00}{\textbf{{#1}}}}
\newcommand{\NormalTok}[1]{{#1}}
\usepackage{longtable,booktabs}
\usepackage{caption}
% These lines are needed to make table captions work with longtable:
\makeatletter
\def\fnum@table{\tablename~\thetable}
\makeatother
\usepackage{graphicx,grffile}
\makeatletter
\def\maxwidth{\ifdim\Gin@nat@width>\linewidth\linewidth\else\Gin@nat@width\fi}
\def\maxheight{\ifdim\Gin@nat@height>\textheight0.8\textheight\else\Gin@nat@height\fi}
\makeatother
% Scale images if necessary, so that they will not overflow the page
% margins by default, and it is still possible to overwrite the defaults
% using explicit options in \includegraphics[width, height, ...]{}
\setkeys{Gin}{width=\maxwidth,height=\maxheight,keepaspectratio}

% Prevent slide breaks in the middle of a paragraph:
\widowpenalties 1 10000
\raggedbottom

\AtBeginPart{
\let\insertpartnumber\relax
\let\partname\relax
\frame{\partpage}
}
\AtBeginSection{
\ifbibliography
\else
\let\insertsectionnumber\relax
\let\sectionname\relax
\frame{\sectionpage}
\fi
}
\AtBeginSubsection{
\let\insertsubsectionnumber\relax
\let\subsectionname\relax
\frame{\subsectionpage}
}

\setlength{\parindent}{0pt}
\setlength{\parskip}{6pt plus 2pt minus 1pt}
\setlength{\emergencystretch}{3em}  % prevent overfull lines
\providecommand{\tightlist}{%
\setlength{\itemsep}{0pt}\setlength{\parskip}{0pt}}
\setcounter{secnumdepth}{0}

\title{Further topics in linear regression}
\subtitle{Part 2}
\author{David Barron}
\date{Hilary Term 2017}

\begin{document}
\frame{\titlepage}

\section{Regression diagnostics}\label{regression-diagnostics}

\subsection{Outliers}\label{outliers}

\begin{frame}{What to look for}

We must identify observations with high \textbf{leverage}; that is, with
an unusual \(x\) value \emph{and} that is out of line with the other
observations. In the figure, the first graph shows an outlier with low
leverage because it is close to the centre of the \(x\) values. The
second graph shows a high leverage outlier. The third graph doesn't
really have an outlier. Although there is one unusual observation, it is
in line with the other cases. Only in the second graph does deletion of
the outlier have much of an impact on the regression line.

\end{frame}

\begin{frame}{High leverage outliers}

\includegraphics{regdiag_fig1.pdf}

\end{frame}

\begin{frame}[fragile]{Example}

\large
Attitudes to inequality

\footnotesize
Data from World Values Survey 1990. \emph{secpay}: attitude to two
secretaries with the same jobs getting paid different amounts if one is
better at the job than the other. 1=Fair, 2=Unfair. Variable is the
national average. \emph{gini}: the gini coefficient of income inequality
in the country. 0=perfect equality, 1=perfect inequality. \emph{gdp}:
GDP per capita in US dollars; \emph{democracy}: 1=experienced democratic
rule for at least 10 years. \textbf{Here we look only at non-democratic
countries}.

\scriptsize

\begin{verbatim}

Call:
lm(formula = secpay ~ gini + gdp, data = weak.nondem)

Residuals:
    Min      1Q  Median      3Q     Max 
-0.1924 -0.0789 -0.0196  0.0382  0.4093 

Coefficients:
              Estimate Std. Error t value    Pr(>|t|)
(Intercept) 1.02826650 0.12778830    8.05 0.000000039
gini        0.00074237 0.00276544    0.27       0.791
gdp         0.00001752 0.00000799    2.19       0.039

Residual standard error: 0.138 on 23 degrees of freedom
Multiple R-squared:  0.175, Adjusted R-squared:  0.104 
F-statistic: 2.45 on 2 and 23 DF,  p-value: 0.109
\end{verbatim}

\end{frame}

\begin{frame}{Scatterplot}

\begin{center}\includegraphics[width=0.9\linewidth]{FurtherTopicsinRegression2_files/figure-beamer/unnamed-chunk-2-1} \end{center}

\end{frame}

\begin{frame}{Hat values}

The hat value is a common way of measuring leverage. Fitted values can
be expressed in terms of observed values: \[
\hat{y}_{j} = h_{1j} y_1 + h_{2j} y_2 + \dots + h_{jj} y_j + \dots + h_{nj} y_n = \sum_{i=1}^n h_{ij} y_i.
\]

So, the weight, \(h_{ij},\) captures the extent to which \(y_i\) can
affect \(\hat{y}_j.\) It may be shown that \(h_i\) summarizes the
potential influence of \(y_i\) on all the fitted values. They are
bounded by \(1/n\) and 1. The average hat-value is \((k+1)/n.\) Values
twice this considered noteworthy (some people use three times).

\end{frame}

\begin{frame}{Hat values plot}

\begin{center}\includegraphics[width=0.9\linewidth]{FurtherTopicsinRegression2_files/figure-beamer/unnamed-chunk-3-1} \end{center}

\footnotesize
Note that there are some cases with bigger hat values that the two
influential cases. Shows limitation of hat values.

\end{frame}

\begin{frame}[fragile]{Studentized residuals}

If we refit the model deleting the \(i\)th observation, we obtain
estimate of the standard deviation of residuals, \(\sigma_{-i}\) based
on \(n-1\) cases.

\[
\epsilon_i^t = \frac{\epsilon_i}{\sigma_{-i} \sqrt{1-h_i}}
\]

Studentized residuals follow a \(t\)-distribution with \(n-k-2\) degrees
of freedom. Observations outside \(\pm 2\) range statistically
significant.

Significance tests have to be corrected for multiple comparisons. This
is done for you using the \texttt{outlierTest} function in the
\texttt{car} package.

\scriptsize

\begin{verbatim}
              rstudent unadjusted p-value Bonferonni p
Slovakia          4.32           0.000278      0.00722
CzechRepublic     2.60           0.016461      0.42798
\end{verbatim}

\end{frame}

\begin{frame}{Studentized residuals plot}

\begin{center}\includegraphics[width=0.9\linewidth]{FurtherTopicsinRegression2_files/figure-beamer/unnamed-chunk-5-1} \end{center}

\end{frame}

\begin{frame}{DFBETA}

A direct measure of the influence of an observation on regression
parameter estimates is: \[
d_{ij} = b_j - b_{j(-i)}
\]

where \(b_{j(-i)}\) is the estimate of \(\beta_j\) with the \(i\)th
observation omitted. These differences are usually scaled by (omitted)
estimates of the standard error of \(b_j\): \[
d_{ij}^* = \frac{d_{ij}}{s_{(-i)}(b_j)}.
\] The \(d_{ij}\) are often termed DFBETA and the \(d_{ij}^*\) are
called DFBETAS.

\end{frame}

\begin{frame}{DFBETA plot}

\begin{center}\includegraphics[width=0.9\linewidth]{FurtherTopicsinRegression2_files/figure-beamer/unnamed-chunk-6-1} \end{center}

\end{frame}

\begin{frame}{Cook's distance}

One way to use DFBETAS is to plot them for each independent variable.
Another is to construct an index. Cook's distance is essentially an
\(F\) statistic for the ``hypothesis'' that
\(\beta_j = b_{j(-i)}, j=0,1,\dots,k.\) This is calculated using:

\[
D_i = \frac{\epsilon^{*2}_i}{k+1} \times \frac{h_i}{1-h_i},
\]

where \(\epsilon^*_i\) is the standardized residual. No formal
hypothesis test, but rule of thumb is \[
D_i > \frac{4}{n-k-1}
\]

\end{frame}

\begin{frame}{Cook's distance plot}

\begin{center}\includegraphics[width=0.9\linewidth]{FurtherTopicsinRegression2_files/figure-beamer/unnamed-chunk-7-1} \end{center}

\end{frame}

\begin{frame}{Rules of thumb}

\begin{itemize}
\tightlist
\item
  \textbf{Hat-values} Values exceeding twice the average \(([k+1]/n)\)
  are noteworthy.
\item
  \textbf{Studentized residuals} About 5\% of these should fall outside
  the range \(|t_i| \le 2.\)
\item
  \textbf{DFBETAS} \(|d_{ij}^*| > 2/\sqrt{n}\)
\item
  \textbf{Cook's D} \(D_i > 4/(n - k - 1).\)
\end{itemize}

\end{frame}

\subsection{Heteroskedasticity}\label{heteroskedasticity}

\begin{frame}{Definition}

Heteroskedasticity occurs when
\(\mathrm{var}(\epsilon_i) \ne \sigma^2,\) but varies across
observations. It is especially problematic when this is related
systematically to an explanatory variable.

\Large Problems

\normalsize

\begin{itemize}
\tightlist
\item
  Increases standard errors of parameter estimates.
\item
  Estimated standard errors are \textbf{biased}.
\end{itemize}

\Large Solutions \normalsize

\begin{itemize}
\tightlist
\item
  Use a different estimator for standard errors.
\item
  Use a different estimator for regression parameters: weighted least
  squares.
\end{itemize}

\end{frame}

\begin{frame}{What to do?}

\begin{itemize}
\tightlist
\item
  Statistical tests

  \begin{itemize}
  \tightlist
  \item
    Goldfeld-Quandt test
  \item
    Breusch-Pagan test
  \end{itemize}
\item
  Remedial action

  \begin{itemize}
  \tightlist
  \item
    Heteroskedasticity-consistent standard errors
  \item
    Weighted least squares
  \end{itemize}
\end{itemize}

\end{frame}

\begin{frame}[fragile]{Director interlocks example}

\tiny
Data on the 248 largest Canadian firms in the mid-1970s.
\emph{interlocks}: the number of board members shared with other major
firms; \emph{assets}: Assets in millions of dollars; \emph{sector}: a
factor with levels BNK=banking, CON=construction, FIN=other financial,
HLD=holding company, MAN=manufacturing, MER=merchandising, MIN=mining,
TRN=transport, WOD=wood and paper; \emph{nation}: nation of control, a
factor with levels CAN=Candian, OTH=Other, UK, US.

\begin{verbatim}

Call:
lm(formula = interlocks ~ I(assets/1000) + sector + nation, data = Ornstein)

Residuals:
   Min     1Q Median     3Q    Max 
-25.00  -6.60  -1.63   4.78  40.73 

Coefficients:
               Estimate Std. Error t value Pr(>|t|)
(Intercept)     10.2669     1.5615    6.58  3.1e-10
I(assets/1000)   0.8096     0.0612   13.23  < 2e-16
sectorBNK      -17.8139     5.9065   -3.02   0.0028
sectorCON       -4.7087     4.7282   -1.00   0.3203
sectorFIN        5.1527     2.6457    1.95   0.0527
sectorHLD        0.8777     4.0041    0.22   0.8267
sectorMAN        1.1487     2.0645    0.56   0.5785
sectorMER        1.4915     2.6359    0.57   0.5721
sectorMIN        4.8803     2.0670    2.36   0.0190
sectorTRN        6.1713     2.7599    2.24   0.0263
sectorWOD        8.2283     2.6786    3.07   0.0024
nationOTH       -1.2413     2.6953   -0.46   0.6456
nationUK        -5.7752     2.6745   -2.16   0.0318
nationUS        -8.6181     1.4963   -5.76  2.6e-08

Residual standard error: 9.83 on 234 degrees of freedom
Multiple R-squared:  0.646, Adjusted R-squared:  0.627 
F-statistic: 32.9 on 13 and 234 DF,  p-value: <2e-16
\end{verbatim}

\end{frame}

\begin{frame}{Heteroskedastic errors}

A common diagnostic is to plot studentised residuals against fitted
values. The cone shape is characteristic of heteroskedastic errors.

\begin{center}\includegraphics[width=0.9\linewidth]{FurtherTopicsinRegression2_files/figure-beamer/unnamed-chunk-9-1} \end{center}

\end{frame}

\begin{frame}[fragile]{Goldfeld-Quandt test}

\footnotesize
Based on the idea that if the sample observations have been generated
under the conditions of homoscedasticity, then the variance of the
disturbances of one sub-sample is the same as the variances of any other
sub-sample. Order cases by the variable you think variance is associated
with (often fitted values from regression). \[
R = \frac{\text{SSE}_2}{\text{SSE}_1}.
\]

\begin{center}
\begin{tabular}{lr}
  SSE from the 1st regression:&    4245.1\\
  SSE from the 2nd regression: &  17187.4\\
  The $F$-statistic for this test:&    4.04\\
  The $p$-value for this test:     &    $\ll$ 0.05   \\
\end{tabular}
\end{center}

\begin{Shaded}
\begin{Highlighting}[]
\KeywordTok{gqtest}\NormalTok{(inter1, }\DataTypeTok{order.by =} \NormalTok{int1.fit)}
\end{Highlighting}
\end{Shaded}

\begin{verbatim}

    Goldfeld-Quandt test

data:  inter1
GQ = 4, df1 = 100, df2 = 100, p-value = 9e-13
\end{verbatim}

\end{frame}

\begin{frame}[fragile]{Breusch-Pagan test}

Model variances using variables thought to be related to the
heteroskedasticity. First obtain residuals by OLS, then divide these by
an estimate of the variance \(\hat{s}^2\). Use as the dependent
variable, with either the fitted values or some other variable as
``explanatory'' variable; the B-P statistic is the explained variance of
this regression divided by 2. This has a \(\chi^2\) distribution with
degrees of freedom equal to number of regressors in the second
regression.

\begin{Shaded}
\begin{Highlighting}[]
\KeywordTok{ncvTest}\NormalTok{(inter1)}
\end{Highlighting}
\end{Shaded}

\begin{verbatim}
Non-constant Variance Score Test 
Variance formula: ~ fitted.values 
Chisquare = 47    Df = 1     p = 7.15e-12 
\end{verbatim}

\end{frame}

\begin{frame}{Transform standard errors}

A common way of dealing with heteroskedasticity is to transform standard
errors---recall it is standard errors \emph{not} parameter estimates
that are affected by this problem.

\[
V(b) = (X'X)^{-1} X' \text{diag}(e^2) X (X'X)^{-1}.
\]

The variance-covariance matrix of the parameter estimates is transformed
by the square of the residuals. The square root of the diagonal of this
matrix is the standard errors of the parameter estimates. This is very
commonly used in practice now. They are called the
heteroskedasticity-consistent standard errors or robust standard errors.

\end{frame}

\begin{frame}{Example: HCCM results}

The idea is that the HC (``heteroskedasticity-consistent'') standard
errors are used instead of the usual ones to calculate \(t\)-statistics
and hence \(p\)-values. You sometimes see these referred to as
``robust'' standard errors, or ``White-corrected'' standard error (after
their inventor). The method of calculating them is sometimes referred to
as a ``sandwich estimator.''

\scriptsize

\begin{longtable}[]{@{}lcccc@{}}
\toprule
& Estimate & H-C Std. Error & t-value &
Pr(\textgreater{}\textbar{}t\textbar{})\tabularnewline
\midrule
\endhead
(Intercept) & 10.27 & 1.50 & 6.83 & 0.00\tabularnewline
I(assets/1000) & 0.81 & 0.09 & 9.32 & 0.00\tabularnewline
sectorBNK & -17.81 & 5.10 & -3.50 & 0.00\tabularnewline
sectorCON & -4.71 & 2.68 & -1.76 & 0.08\tabularnewline
sectorFIN & 5.15 & 2.70 & 1.91 & 0.06\tabularnewline
sectorHLD & 0.88 & 4.47 & 0.20 & 0.84\tabularnewline
sectorMAN & 1.15 & 1.74 & 0.66 & 0.51\tabularnewline
sectorMER & 1.49 & 2.20 & 0.68 & 0.50\tabularnewline
sectorMIN & 4.88 & 1.81 & 2.70 & 0.01\tabularnewline
sectorTRN & 6.17 & 3.07 & 2.01 & 0.04\tabularnewline
sectorWOD & 8.23 & 3.27 & 2.51 & 0.01\tabularnewline
nationOTH & -1.24 & 2.81 & -0.44 & 0.66\tabularnewline
nationUK & -5.78 & 2.06 & -2.81 & 0.00\tabularnewline
nationUS & -8.62 & 1.38 & -6.25 & 0.00\tabularnewline
\bottomrule
\end{longtable}

\end{frame}

\subsection{Linearity}\label{linearity}

The standard model we've been looking at involves linear relationships
between explanatory and outcome variables, i.e., the effect of a change
in \(X\) is the same at all values of \(X\). This may not always be the
case. For example, wages typically increase with age up to a point and
then start to decline again. Although it is preferable to make decisions
based on theory, it is possible to use graphical methods to check
linearity assumptions.

The following exampmle uses data from the Candadian Survey of Labour and
Income Dynamics (1994).

\begin{verbatim}

Call:
lm(formula = Logwages ~ education + age + sex + language, data = SLID)

Residuals:
    Min      1Q  Median      3Q     Max 
-2.3646 -0.2768  0.0146  0.2844  1.5667 

Coefficients:
               Estimate Std. Error t value Pr(>|t|)
(Intercept)    1.118413   0.038827   28.81   <2e-16
education      0.055035   0.002204   24.97   <2e-16
age            0.017621   0.000553   31.89   <2e-16
sexMale        0.224259   0.013266   16.90   <2e-16
languageFrench 0.004922   0.027061    0.18     0.86
languageOther  0.009927   0.020614    0.48     0.63

Residual standard error: 0.419 on 3981 degrees of freedom
Multiple R-squared:  0.309, Adjusted R-squared:  0.309 
F-statistic:  357 on 5 and 3981 DF,  p-value: <2e-16
\end{verbatim}

\begin{frame}{Residual plots}

The most straightforward thing to do is plot residuals against each of
the explanatory variables to look for evidence of non-linearity.

\includegraphics[width=0.45\linewidth]{FurtherTopicsinRegression2_files/figure-beamer/unnamed-chunk-14-1}
\includegraphics[width=0.45\linewidth]{FurtherTopicsinRegression2_files/figure-beamer/unnamed-chunk-14-2}

\begin{longtable}[]{@{}lrr@{}}
\toprule
& Education & Age\tabularnewline
\midrule
\endhead
Test & 3.66 & -21.4\tabularnewline
Pvalue & 0.00 & 0.0\tabularnewline
\bottomrule
\end{longtable}

This function also provides a test of whether adding a quadratic term
would be statistically significant.

\end{frame}

\begin{frame}{Component plus residual plots}

The y axis is: \[
e + \hat{\beta}_i X_i
\] where \(e\) are residuals, \(\hat{\beta}_i\) is the estimated
regression parameter for the \(i\)th explanatory variable, \(X_i\),
which is plotted on the x-axis. The augmented plots shown also have the
linear fit (red dotted line) and a non-parametric `smoother' (green
solid line). This can also show a departure from linearity.

\includegraphics[width=0.45\linewidth]{FurtherTopicsinRegression2_files/figure-beamer/unnamed-chunk-15-1}
\includegraphics[width=0.45\linewidth]{FurtherTopicsinRegression2_files/figure-beamer/unnamed-chunk-15-2}

\end{frame}

\begin{frame}[fragile]{Add quadratic terms}

\tiny

\begin{verbatim}

Call:
lm(formula = Logwages ~ sex + language + poly(education, 2, raw = TRUE) + 
    poly(age, 2, raw = TRUE), data = SLID)

Residuals:
    Min      1Q  Median      3Q     Max 
-2.0855 -0.2404  0.0223  0.2515  1.7813 

Coefficients:
                                  Estimate Std. Error t value  Pr(>|t|)
(Intercept)                      0.4055703  0.0920226    4.41 0.0000107
sexMale                          0.2215223  0.0125469   17.66   < 2e-16
languageFrench                  -0.0133857  0.0255896   -0.52      0.60
languageOther                   -0.0089290  0.0197519   -0.45      0.65
poly(education, 2, raw = TRUE)1 -0.0023108  0.0110123   -0.21      0.83
poly(education, 2, raw = TRUE)2  0.0018456  0.0004094    4.51 0.0000067
poly(age, 2, raw = TRUE)1        0.0835514  0.0031156   26.82   < 2e-16
poly(age, 2, raw = TRUE)2       -0.0008590  0.0000398  -21.57   < 2e-16

Residual standard error: 0.395 on 3979 degrees of freedom
Multiple R-squared:  0.384, Adjusted R-squared:  0.383 
F-statistic:  354 on 7 and 3979 DF,  p-value: <2e-16
\end{verbatim}

\begin{longtable}[]{@{}rrrrrr@{}}
\toprule
Res.Df & RSS & Df & Sum of Sq & F & Pr(\textgreater{}F)\tabularnewline
\midrule
\endhead
3981 & 697 & NA & NA & NA & NA\tabularnewline
3979 & 622 & 2 & 75.1 & 240 & 0\tabularnewline
\bottomrule
\end{longtable}

\end{frame}

\begin{frame}{Effect plots}

\includegraphics[width=0.45\linewidth]{FurtherTopicsinRegression2_files/figure-beamer/unnamed-chunk-17-1}
\includegraphics[width=0.45\linewidth]{FurtherTopicsinRegression2_files/figure-beamer/unnamed-chunk-17-2}

\end{frame}

\section{Selection models}\label{selection-models}

\begin{frame}{Sample selection bias}

A general issue, not only concerning linear regression. It is important
because it undermines external and internal validity. That is, the
problem is not solved by claiming to be interested only in a sub-set of
the population. In effect sample selection excludes a regressor that is
correlated with an included regressor.

\end{frame}

\begin{frame}{Illustration}

\includegraphics{SampleSelectionBias.pdf}

\end{frame}

\begin{frame}{Types of selection}

\includegraphics{TypesSelection.pdf}

\end{frame}

\begin{frame}{Intuition}

\begin{itemize}
\tightlist
\item
  Non-random selection---inference may not extend to the unobserved
  group
\item
  Example: Suppose we observe that college grades are uncorrelated with
  success in graduate school
\item
  Can we infer that college grades are irrelevant?
\item
  No: applicants admitted with low grades may not be representative of
  the population with low grades
\item
  Unmeasured variables (e.g.~motivation) used in the admissions process
  might explain why those who enter graduate school with low grades do
  as well as those who enter graduate school with high grades
\end{itemize}

\end{frame}

\begin{frame}{Selection equation}

\begin{itemize}
\tightlist
\item
  \(z^*_i =\)latent variable, DV of selection equation; the propensity
  to be including in the sample;
\item
  \(w'_i =\) vector of covariates for unit \(i\) for selection equation;
\item
  \(\alpha =\) vector of coefficients for selection equation;
\item
  \(\epsilon_i =\) random disturbance for unit \(i\) for selection
  equation;
\end{itemize}

\[
z^*_i = w'_i \alpha + \epsilon_i.
\]

\Large Outcome equation

\normalsize

\begin{itemize}
\tightlist
\item
  \(y_i =\) DV from outcome equation;
\item
  \(x'_i =\) vector of covariates for unit \(i\) for outcome equation;
\item
  \(\beta =\) vector of coefficients for outcome equation;
\item
  \(u_i =\) random disturbance for unit \(i\) for outcome equation;
\end{itemize}

\end{frame}

\begin{frame}{Heckman model}

Assume that \(y_i\) is observed if and only if a second, unobserved
latent variable, \(z^*_i\) exceeds a particular threshold:

\[
z_i = \left\{ \begin{aligned}
  1 \text{ if } z^*_i > 0; \\
  0 \text{ otherwise}
\end{aligned}\right.
\]

So, we first estimate the probability that \(z_i = 1,\) and use a
transformation of this predicted probability as an independent variable
in the outcome equation.

\end{frame}

\begin{frame}{Sample selection bias: Conclusions}

\begin{itemize}
\tightlist
\item
  If potential observations from some population of interest are
  excluded on a nonrandom basis, one risks sample selection bias.
\item
  It is difficult to anticipate whether the biased regression estimates
  overstate or understate the true causal effects.
\item
  Problems caused by nonrandom exclusion of observations are manifested
  in the expected values of the endogenous variable.
\end{itemize}

\end{frame}

\begin{frame}[fragile]{Example}

\scriptsize
A common example of sample selection is when studying wages. In order to
earn a wage, you have to have a job. You are more likely to have a job
if you are able to earn a good wage. So, there is likely to be sample
selection. This is the ordinary regression.

\begin{verbatim}

Call:
lm(formula = Logwage ~ education + age, data = wom)

Residuals:
    Min      1Q  Median      3Q     Max 
-1.3116 -0.1389  0.0223  0.1769  0.6962 

Coefficients:
            Estimate Std. Error t value Pr(>|t|)
(Intercept) 2.362865   0.041026   57.59   <2e-16
education   0.038897   0.002297   16.93   <2e-16
age         0.006358   0.000863    7.37    3e-13

Residual standard error: 0.251 on 1340 degrees of freedom
  (657 observations deleted due to missingness)
Multiple R-squared:  0.231, Adjusted R-squared:  0.23 
F-statistic:  201 on 2 and 1340 DF,  p-value: <2e-16
\end{verbatim}

\end{frame}

\begin{frame}[fragile]{Sample selection results}

\scriptsize

\begin{verbatim}
--------------------------------------------
Tobit 2 model (sample selection model)
Maximum Likelihood estimation
Newton-Raphson maximisation, 5 iterations
Return code 2: successive function values within tolerance limit
Log-Likelihood: -1057 
2000 observations (657 censored and 1343 observed)
10 free parameters (df = 1990)
Probit selection equation:
            Estimate Std. error t value Pr(> t)
(Intercept) -2.48770    0.19248  -12.92 < 2e-16
married      0.46293    0.07203    6.43 1.3e-10
children     0.47002    0.02808   16.74 < 2e-16
education    0.05465    0.01096    4.99 6.1e-07
age          0.03504    0.00423    8.29 < 2e-16
Outcome equation:
            Estimate Std. error t value Pr(> t)
(Intercept)  2.19621    0.04761   46.13  <2e-16
education    0.04167    0.00238   17.51  <2e-16
age          0.00836    0.00092    9.09  <2e-16
   Error terms:
      Estimate Std. error t value Pr(> t)
sigma  0.26192    0.00611   42.88  <2e-16
rho    0.47822    0.05768    8.29  <2e-16
--------------------------------------------
\end{verbatim}

\end{frame}

\end{document}
